\section{Troubleshooting}\label{sect:troubleshooting}

Sollten w�hrend der Installation oder der Nutzung von \ATFROGS Probleme auftreten, sind die folgenden Schritte grunds�tzlich durchzuf�hren:

\begin{itemize}
\item Analysieren der Fehlermeldung innerhalb von \ATFROGS
\item Einsehen der \ATFROGSnosp-Log-Dateien
\item Einsehen der Tomcat-Log-Dateien
\end{itemize} 

\NOTE{Ist das Attribut \texttt{logLevel} in der \texttt{web.xml} auf \texttt{off}, so werden keine Meldungen in die \ATFROGSnosp-Log-Dateien geschrieben. Im Level \texttt{debug} ist der Umfang der geschriebenen Informationen am gr��ten (vgl. Abschnitt \ref{sect:install:editweb.xml}).}\vspace{5mm}

Im Folgenden werden h�ufig autretende Fehler und L�sungsans�tze aufgef�hrt.\


\subsection*{Kein Zugriff auf den Tomcat-Manager}
\textbf{Problembeschreibung:} �ber \url{http://[myOXServer]:8080}\ ist kein Zugriff auf den Tomcat-Manager m�glich.\\
\textbf{M�gliche L�sung: } �berpr�fen Sie die Einstellungen der Server-Firewall bez�glich des Ports 8080. Zumindest vom Server aus sollte ein Zugriff �ber \url{http://localhost:8080}\ m�glich sein. \\

\subsection*{Web-Applikation nicht im Tomcat-Manager}
\textbf{Problembeschreibung:} \ATFROGS erscheint nach der Installation der Web-Applikation durch den Tomcat-Manager nicht in dessen Liste der installierten Anwendungen.\\
\textbf{M�gliche L�sung: } Der Benutzer \textit{tomcat} verf�gt nicht �ber die entsprechenden Rechte im Ordner \textit{webapps} des Tomcat-Verzeichnisses. Mit \texttt{chown -R tomcat webapps/} sollte dies ge�ndert werden.\\

\subsection*{Anmeldung an \ATFROGS nicht m�glich}
\textbf{Problembeschreibung:} Die Login-Seite von \ATFROGS erscheint, jedoch ist das Anmelden nicht m�glich. \\
\textbf{M�gliche L�sung:} Der einfachste Fall ist, dass Benutzername oder Kennwort falsch sind. Ist dies nicht der Fall, wurde die Authentifikation in der \texttt{server.xml} m�glicherweise fehlerhaft eingestellt (vgl. Abschnitt \ref{sect:install:editserver.xml}). Weitere Informationen k�nnte die \texttt{localhost}-Log-Datei des Tomcat liefern. Stellen Sie au�erdem sicher, dass der Benutzer der Gruppe der \ATFROGSnosp-Administratoren zugeordnet ist. Sollte das Anmelden trotzdem nicht m�glich sein, so k�nnte das Kennwort des Benutzers m�glicherweise nicht mit dem in der \texttt{server.xml} angegebenen Mechanismus verschl�sselt sein. Loggen Sie sich dazu mit dem entsprechenden Benutzer in der Groupware ein und �ndern Sie den Verschl�sselungsmechanismus entsprechend.

\subsection*{Fehlermeldung nach dem Anmelden}
\textbf{Problembeschreibung:} Das Anmelden an \ATFROGS ist erfolgreich, jedoch wird statt eines \ATFROGSnosp-Moduls eine Meldung eines schwer wiegenden Fehlers angezeigt. \\
\textbf{M�gliche L�sung:} In diesem Fall sollten vor allem die Log-Datei des Tomcat eingesehen werden. M�glicherweise sind die in der \texttt{web.xml} eingetragenen Daten fehlerhaft (Benutzername oder Kennwort des dort eingetragenen Benutzers) oder der Benutzer ist in keiner \OPENXCHANGE-Gruppe. Stellen Sie au�erdem sicher, dass die \OPENXCHANGE-WebDAV-Schnittstelle korrekt funktioniert.\\

\subsection*{Logger kann nicht initialisiert werden}
\textbf{Problembeschreibung:} Das Anmelden an \ATFROGS ist erfolgreich, jedoch wird statt eines \ATFROGSnosp-Moduls die Meldung \textit{cannot initiate the logger} angezeigt. \\
\textbf{M�gliche L�sung:} Der Benutzer \textit{tomcat} verf�gt nicht �ber die entsprechenden Berechtigungen im Verzeichnis der \ATFROGSnosp-Log-Dateien oder ein solches Verzeichnis existiert nicht. �berpr�fen Sie die von Ihnen durchgef�hrten Schritte im Abschnitt \ref{sect:install:editweb.xml}\\
