\appendix

\section{Anhang}\label{sect:anhang}

Die folgenden beiden Abschnitte \ref{server.inc.xml} und \ref{web.inc.xml} enthalten die Listings der beiden Konfigurationsdateien \textit{server.xml} und \textit{web.xml}, die gem�� Abschnitt \ref{sect:install:konfigurationsdateien} anzupassen sind. Vollst�ndige Musterdateien finden sich in dem Verzeichnis, in dem sich auch dieses Dokument innerhalb des Archivs \textit{atfrogs-<version>.tar.gz} befindet. 

\subsection{Ausschnitt \textit{server.xml}}\label{server.inc.xml}
Das folgende Listing enh�lt einen Ausschnitt aus der Konfigurationsdatei \textit{server.xml} mit beispielhaften Einstellungen f�r die Authentifizierung der \ATFROGSnosp-Benutzer gegen den (auch von \OPENXCHANGEsp benutzten) LDAP-Server.\\

\begin{small}
\lstset{numbers=left, numberstyle=\tiny, stepnumber=2, numbersep=5pt, language=xml}
\begin{lstlisting}[caption=Ausschnitt server.xml, captionpos=b, frame=single, breaklines=true, firstnumber=373, numberfirstline=false]
        <Logger className="org.apache.catalina.logger.FileLogger"
         directory="logs"  prefix="localhost_log." suffix=".txt"
         timestamp="true"/>

        <Context path="/ATFrogs" docBase="ATFrogs" debug="9" 
         reloadable="true" >
          <Realm className="org.apache.catalina.realm.JNDIRealm"
           debug="99"
           connectionName="cn=Manager,dc=YOUR-COMPANY,dc=de"
           connectionPassword="YOUR-PASSWORD"
           connectionURL="ldap://GROUPWARE.YOUR-COMPANY:389"
           userSearch="(uid={1})"
           userPassword="userPassword"
           userPattern="uid={0},ou=Users,ou=OxObjects,dc=your-company,dc=de"
           roleBase="ou=Groups,ou=OxObjects,dc=your-company,dc=de"
           roleName="cn"
           roleSearch="(memberUid={1})"
           roleSubtree="false"
           digest="SHA"/>
        </Context>

      </Host>

    </Engine>

  </Service>

</Server>
\end{lstlisting}
\end{small}

\pagebreak

\subsection{Ausschnitt \textit{web.xml}}\label{web.inc.xml}
Das folgende Listing enh�lt einen Ausschnitt aus der Konfigurationsdatei \textit{web.xml} mit den wichtigsten Einstellungen.\\

\begin{small}
\lstset{numbers=left, numberstyle=\tiny, stepnumber=2, numbersep=5pt, language=xml}
\begin{lstlisting}[caption=Ausschnitt web.xml, captionpos=b, frame=single, breaklines=true, firstnumber=80]
<servlet>
  <servlet-name>ATFrogs</servlet-name>
  <servlet-class>atfrogs.servlets.ATFrogsMainServlet</servlet-class>
  <init-param>
    <param-name>user</param-name>
    <param-value>mailadmin</param-value>
  </init-param>
  <init-param>
    <param-name>pass</param-name>
    <param-value>YOUR-PASSWORD</param-value>
  </init-param>
  <init-param>
    <param-name>servleturl</param-name>
    <param-value>http://localhost/servlet/webdav.groupuser</param-value>
  </init-param>
  <init-param>
    <!-- tomcat user must of write access to directory
         tip: create directory for these log files -->
    <param-name>logfile</param-name>
    <param-value>/var/log/ATFrogs/ATFrogs.log</param-value>
  </init-param>
  <init-param>
    <!-- To specify level of logging. Possible values are:
         off    - no logging
         normal - logging of user actions (default)
         debug  - additional messages for debugging
         Note: Also concerns output to catalina.out -->
    <param-name>logLevel</param-name>
    <param-value>debug</param-value>
  </init-param>	
  <init-param>
    <!-- categorie to show after login {groups | resources | user } -->
    <param-name>defaultCategorie</param-name>
    <param-value>groups</param-value>
  </init-param>	
  <init-param>
    <!-- language {en | de} -->
    <param-name>language</param-name>
    <param-value>de</param-value>
  </init-param>
</servlet>
\end{lstlisting}
\end{small}
