\section{Allgemeines}

\subsection{Vorwort}
\ATFROGS ist eine Administrationsoberfl�che f�r \OPENXCHANGEsp und wurde im Rahmen des Seminars \glqq das Ph�nomen Open Source -- interdisziplin�r\grqq\ an der Carl von Ossietzky Universit�t Oldenburg entwickelt.\footnote{Unter der Adresse \texttt{http://atfrogs.berlios.de/\#download} ist die jeweils aktuelle \ATFROGSnosp-Version abrufbar.} \OPENXCHANGEsp ist eine freie Groupware-Software der Netline Internet Service GmbH. \\

Durch die komfortable Oberfl�che von \ATFROGS lassen sich Gruppen, Ressourcen und Benutzer administrieren, ohne die Administrationsskripte der \OPENXCHANGE-Installation von Hand aufrufen zu m�ssen.\footnote{Neben den Administrationsskripten der \OPENXCHANGE-Installation enth�lt \ATFROGS das Skript \textit{resetuserpwd}, das f�r die Funktion zum Zur�cksetzen von Benutzerkennw�rtern ben�tigt wird (vgl. Abschnitt \ref{sect:useradministration:resetuserpwd}).} Benutzeroberfl�che und Bedienung sind \OPENXCHANGEsp nachempfunden, was eine Nutzung ohne gr��ere Einarbeitung erm�glicht. Eine �bersicht der von \ATFROGS gebotenen Funktionalit�t findet sich im folgenden Abschnitt \ref{sect:allg:funktionsumgang}.\\

Die Gew�hrung von Zugriff zu \ATFROGS erfolgt gruppenbasiert. Ausschlie�lich die Benutzer, die einer dedizierten \OPENXCHANGE-Gruppe zugeordnet sind, erhalten Zugang zu \ATFROGSnosp. \\

Sowohl \ATFROGS als auch \OPENXCHANGEsp sind lizenziert unter der General Public License\footnote{Der englischsprachige Lizenztext findet sich unter \url{http://www.fsf.org/licensing/licenses/gpl.html}.} der Free Software Foundation. \\

Im vorliegenden Handbuch wird in Kapitel \ref{sect:install} eine detaillierte Anleitung gegeben, wie \ATFROGS auf einem bestehenden \OPENXCHANGE-Server einzurichten ist. Die sich daran anschlie�enden Kapitel geben Hinweise zur Benutzung der einzelnen \ATFROGSnosp-Module. Im Kapitel \ref{sect:troubleshooting} finden sich Hinweise f�r die F�lle, dass es w�hrend der Installation oder des Betriebs von \ATFROGS zu Problemen kommt.

\subsection{Funktionsumfang}\label{sect:allg:funktionsumgang} 

\ATFROGS bietet folgende Funktionen zur \underline{Gruppenverwaltung}:
\begin{itemize}
\item �bersicht aller bestehender Gruppen
\item Detailansicht einer Grupppe mit allen Gruppenmitgliedern
\item Hinzuf�gen von Benutzern zu Gruppen
\item Entfernen von Benutzern aus Gruppen
\item Erstellen neuer Gruppen
\item L�schen bestehender Gruppen
%Aufl�sen von Gruppen und Suchen nach Benutzern rausgelassen
\end{itemize}\
 
\pagebreak

\ATFROGS bietet folgende Funktionen zur \underline{Ressourcenverwaltung}:
\begin{itemize}
\item �bersicht aller bestehender Ressourcen und Ressourcengruppen
\item Detailansicht einer Ressourcengruppe mit allen, der Gruppe zugeordneten Ressourcen
\item Hinzuf�gen von Ressourcen zu Ressourcengruppen
\item Entfernen von Ressourcen aus Ressourcengruppen
\item Erstellen neuer Ressourcen und Ressourcengruppen
\item L�schen bestehender Ressourcen und Ressourcengruppen
% Suche von Ressourcen �ber Suchmuster
% Aufl�sen von Ressourcengruppen
\end{itemize}\

\ATFROGS bietet folgende Funktionen zur \underline{Benutzerverwaltung}:
\begin{itemize}
\item Suchen von Benutzern �ber ein Suchmuster
\item Zur�cksetzen von Kennw�rtern
\item L�schen von Benutzern
\end{itemize}\

Des Weiteren bietet \ATFROGS folgende Funktionen:
\begin{itemize}
\item LDAP-Authentifizierung basierend auf \OPENXCHANGE-Gruppen
\item Anzeige der Konsolenausgabe nach Ausf�hren eines Skripts in einem Textfenster
\item Aufzeichnung aller Benutzeraktionen mit Benutzername, Datum und Zeit
\item Sprachunterst�tzung englisch und deutsch
\item Einfache Erweiterbarkeit auf weitere Sprachen m�glich.
\end{itemize}\

\subsection{Systemvoraussetzungen}
\ATFROGS ist eine sogenannte Java-Web-Applikation, die auf einem Tomcat-Server zu installieren ist. Dies sollte der Server sein, auf dem \OPENXCHANGEsp ab Version 0.8 installiert ist. Die \OPENXCHANGE-WebDAV-Schnittstelle muss aktiviert und funktionsf�hig sein.\\

\subsection{Schriftformen und Hervorhebungen}

Innerhalb dieses Dokuments werden bestimmte Informationen durch eine, sich abhebende, Schriftform oder eine anderweitige Hervorhebung vom Text abgesetzt. Die verschiedenen Typen und deren Bedeutungen sind der folgenden Auflistung zu entnehmen:

\begin{itemize}
\item Neben Dateinamen und Pfaden werden Beschriftungen von Schaltfl�chen, Gruppen-, Benutzer- sowie Firmennamen \textit{kursiv} dargestellt. 
\item F�r Internetadressen, auszuf�hrende Kommandos, Parameter in Konfigurationsdateien und zu bet�tigende Tasten wird eine \texttt{nicht-proportionale} Schriftform verwendet.
\item Zu ersetzende Teile von Zeichenfolgen sind von eckigen Klammern umgeben (zum Beispiel 
\url{http://[ihrOXServer.de]:8080/ATFrogs}). Bei Versionsnummern finden spitze Klammern Verwendung (zum Beispiel \textit{atfrogs-<version>.tar.gz}).
\item Wichtige Hinweise werden innerhalb eines roten K�stchens in {\sffamily serifenfreier} Schrift hervorgehoben. 
\item Schaltfl�chen, die eine Grafik enthalten, werden mit deren Symbol, so wie sie innerhalb der Oberfl�che von \ATFROGS erscheinen, dargestellt (zum Beispiel \ICON{graphics/item_new.png}).
\end{itemize}



